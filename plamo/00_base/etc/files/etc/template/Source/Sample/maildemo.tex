%% この tex ファイルをコンパイルするためには、linux-pengo-colors.gif から 
%% EPSF ファイルを作らなければなりません。そのためには、netpbm パッケージの
%% giftopnm と pnmtops コマンドを使います。シェルプロンプトから、
%% $ giftopnm linux-pengo-colors.gif | pnmtops > linux-pengo-colors.eps
%% コマンドを実行して、GIF ファイルを EPSF ファイルに変換してください。


\documentclass{jarticle}
\usepackage[dvips]{graphicx}

\begin{document}

\begin{center}
{\Large \bf Plamo Linux へようこそ!!}
\end{center}

Plamo Linux は、日本人による日本人のための日本語環境を追及した 
Linux パッケージです。

このメールは MIME エンコードされた画像データを含んでいます。mule + Mew 
で読んでいる場合、space キーを押していけば、自動的に画像データも外部のビュー
ワ(xv)を使って表示されるはずです。また、このメールには、同じ内容を TeX +
dvi2ps で PostScript 化したパートも添付しています。gv と ghostscript が正
しくインストールできていれば、PS ファイルの部分も読めるはずです。

Plamo Linux は Slackware 3.3 を元に、PJE や JF, JG といった日本の Linux 
関係者の努力の成果を集めたパッケージです。Plagiaware の作者はそれらのプ
ロジェクトの成果をいただいてまとめ直しただけです。もし、このパッケージが
多少なりとも役に立つようでしたら、PJE や JE、JF の関係者の努力のおかげと
お考えください。

作者の力不足で、Plamo Linux にはまだまだ不完全な部分があると思います。問
題点やここはこう改良した方がいいという意見はどんなものでも結構ですから、
作者(kojima@linet.gr.jp)までお寄せください。

Plamo Linux で、少しでも Linux の日本語環境が便利に使えますように。。

\includegraphics[scale=1.5,angle=-90]{linux-pengo-colors.eps}
\begin{flushright}
\begin{tabular}{l}
1998/08/20   こじまみつひろ \\
kojima@linet.gr.jp(home) \\)
kojima@criepi.denken.or.jp(office) \\
\end{tabular}
\end{flushright}



\end{document}
